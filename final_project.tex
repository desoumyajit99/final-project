% Options for packages loaded elsewhere
% Options for packages loaded elsewhere
\PassOptionsToPackage{unicode}{hyperref}
\PassOptionsToPackage{hyphens}{url}
\PassOptionsToPackage{dvipsnames,svgnames,x11names}{xcolor}
%
\documentclass[
  stu,
  floatsintext]{apa7}
\usepackage{xcolor}
\usepackage{amsmath,amssymb}
\setcounter{secnumdepth}{-\maxdimen} % remove section numbering
\usepackage{iftex}
\ifPDFTeX
  \usepackage[T1]{fontenc}
  \usepackage[utf8]{inputenc}
  \usepackage{textcomp} % provide euro and other symbols
\else % if luatex or xetex
  \usepackage{unicode-math} % this also loads fontspec
  \defaultfontfeatures{Scale=MatchLowercase}
  \defaultfontfeatures[\rmfamily]{Ligatures=TeX,Scale=1}
\fi
\usepackage{lmodern}
\ifPDFTeX\else
  % xetex/luatex font selection
\fi
% Use upquote if available, for straight quotes in verbatim environments
\IfFileExists{upquote.sty}{\usepackage{upquote}}{}
\IfFileExists{microtype.sty}{% use microtype if available
  \usepackage[]{microtype}
  \UseMicrotypeSet[protrusion]{basicmath} % disable protrusion for tt fonts
}{}
\makeatletter
\@ifundefined{KOMAClassName}{% if non-KOMA class
  \IfFileExists{parskip.sty}{%
    \usepackage{parskip}
  }{% else
    \setlength{\parindent}{0pt}
    \setlength{\parskip}{6pt plus 2pt minus 1pt}}
}{% if KOMA class
  \KOMAoptions{parskip=half}}
\makeatother
% Make \paragraph and \subparagraph free-standing
\makeatletter
\ifx\paragraph\undefined\else
  \let\oldparagraph\paragraph
  \renewcommand{\paragraph}{
    \@ifstar
      \xxxParagraphStar
      \xxxParagraphNoStar
  }
  \newcommand{\xxxParagraphStar}[1]{\oldparagraph*{#1}\mbox{}}
  \newcommand{\xxxParagraphNoStar}[1]{\oldparagraph{#1}\mbox{}}
\fi
\ifx\subparagraph\undefined\else
  \let\oldsubparagraph\subparagraph
  \renewcommand{\subparagraph}{
    \@ifstar
      \xxxSubParagraphStar
      \xxxSubParagraphNoStar
  }
  \newcommand{\xxxSubParagraphStar}[1]{\oldsubparagraph*{#1}\mbox{}}
  \newcommand{\xxxSubParagraphNoStar}[1]{\oldsubparagraph{#1}\mbox{}}
\fi
\makeatother


\usepackage{longtable,booktabs,array}
\usepackage{calc} % for calculating minipage widths
% Correct order of tables after \paragraph or \subparagraph
\usepackage{etoolbox}
\makeatletter
\patchcmd\longtable{\par}{\if@noskipsec\mbox{}\fi\par}{}{}
\makeatother
% Allow footnotes in longtable head/foot
\IfFileExists{footnotehyper.sty}{\usepackage{footnotehyper}}{\usepackage{footnote}}
\makesavenoteenv{longtable}
\usepackage{graphicx}
\makeatletter
\newsavebox\pandoc@box
\newcommand*\pandocbounded[1]{% scales image to fit in text height/width
  \sbox\pandoc@box{#1}%
  \Gscale@div\@tempa{\textheight}{\dimexpr\ht\pandoc@box+\dp\pandoc@box\relax}%
  \Gscale@div\@tempb{\linewidth}{\wd\pandoc@box}%
  \ifdim\@tempb\p@<\@tempa\p@\let\@tempa\@tempb\fi% select the smaller of both
  \ifdim\@tempa\p@<\p@\scalebox{\@tempa}{\usebox\pandoc@box}%
  \else\usebox{\pandoc@box}%
  \fi%
}
% Set default figure placement to htbp
\def\fps@figure{htbp}
\makeatother





\setlength{\emergencystretch}{3em} % prevent overfull lines

\providecommand{\tightlist}{%
  \setlength{\itemsep}{0pt}\setlength{\parskip}{0pt}}



 


\usepackage[american]{babel}
\usepackage{csquotes}
\usepackage{amsmath}
\usepackage{bm}
\usepackage{float}
\shorttitle{Final Project Poverty}
\affiliation{\mbox{Department of Communication, University of California, Santa Barbara}}
\course{PSTAT 220A Advanced Statistical Methods}
\professor{Instructor: Dr. Alexander Franks}
\duedate{December 8, 2025}
\setlength{\parindent}{0.5in}
\setlength{\parskip}{0in}
\makeatletter
\@ifpackageloaded{caption}{}{\usepackage{caption}}
\AtBeginDocument{%
\ifdefined\contentsname
  \renewcommand*\contentsname{Table of contents}
\else
  \newcommand\contentsname{Table of contents}
\fi
\ifdefined\listfigurename
  \renewcommand*\listfigurename{List of Figures}
\else
  \newcommand\listfigurename{List of Figures}
\fi
\ifdefined\listtablename
  \renewcommand*\listtablename{List of Tables}
\else
  \newcommand\listtablename{List of Tables}
\fi
\ifdefined\figurename
  \renewcommand*\figurename{Figure}
\else
  \newcommand\figurename{Figure}
\fi
\ifdefined\tablename
  \renewcommand*\tablename{Table}
\else
  \newcommand\tablename{Table}
\fi
}
\@ifpackageloaded{float}{}{\usepackage{float}}
\floatstyle{ruled}
\@ifundefined{c@chapter}{\newfloat{codelisting}{h}{lop}}{\newfloat{codelisting}{h}{lop}[chapter]}
\floatname{codelisting}{Listing}
\newcommand*\listoflistings{\listof{codelisting}{List of Listings}}
\makeatother
\makeatletter
\makeatother
\makeatletter
\@ifpackageloaded{caption}{}{\usepackage{caption}}
\@ifpackageloaded{subcaption}{}{\usepackage{subcaption}}
\makeatother
\usepackage{bookmark}
\IfFileExists{xurl.sty}{\usepackage{xurl}}{} % add URL line breaks if available
\urlstyle{same}
\hypersetup{
  pdftitle={Final Project: Poverty and County-Level Demographic, Socioeconomic, and Geographic Estimates},
  pdfauthor={Soumyajit De},
  colorlinks=true,
  linkcolor={blue},
  filecolor={Maroon},
  citecolor={Blue},
  urlcolor={Blue},
  pdfcreator={LaTeX via pandoc}}


\title{Final Project: Poverty and County-Level Demographic,
Socioeconomic, and Geographic Estimates}
\author{Soumyajit De}
\date{}
\begin{document}
\maketitle


In this project, I analyze county-level patterns of poverty in the
United States using a cross-sectional dataset of 3,141 counties drawn
from all 50 states. Each observation corresponds to a single county and
includes the percentage of residents living below the poverty line,
overall population size, and a range of demographic, socioeconomic, and
geographic characteristics. Poverty is defined as the percentage of
residents whose income falls below the official poverty threshold for a
given year. It serves as the primary outcome variable for all analyses.
The first research question examines how poverty is related to this
broad set of county-level characteristics. Specifically, I ask:

RQ1. How is county-level poverty (percentage of individuals under the
poverty level) associated with county demographic, geographic, and
socioeconomic characteristics? Which of these variables appear to be
most strongly associated with poverty, and which (if any) show little or
no predictive value?

In addressing this, I group predictor variables into conceptual blocks
(e.g., gender, race/ethnicity, employment sectors, transportation to
work, type of work, economic conditions, education, age, and geography)
and examine their joint associations with county poverty rates. The
second research question narrows the focus to gender composition and
state context:

RQ2. To what extent is county-level poverty associated with the gender
composition of the population in California and Texas, and does this
association differ?

In the following sections, I describe the dataset, key variables, and
preprocessing decisions; present exploratory data analyses to summarize
the distribution of poverty and its associations with the main
predictors; fit multiple linear regression models to address RQ1 and RQ2
using robust standard errors and model diagnostics; apply a
multiple-testing correction to account for the number of predictors
before summarizing the main patterns of association, and and discuss
their implications and limitations.

\section{Methods}\label{methods}

\subsection{Data and Variables}\label{data-and-variables}

For the analyses, I used the \(\texttt{poverty\_data.csv}\) dataset
provided on Canvas. It contains 3,141 cross sectional observations, each
corresponding to a single U.S. county. The primary outcome variable is
Poverty, which I define as the percentage of county residents whose
income falls below the official poverty threshold for a given year. This
variable is expressed on a 0--100 scale and serves as the dependent
variable for both research questions. All remaining variables are
treated as predictors.

Location and geography consists of State, County, and county\_fips,
which identify each county and allow subsetting for RQ2. Long and Lat
provide county-level longitude and latitude that are used to visualize
geographic patterns. The variable TotalPop represents the total number
of residents in each county. Gender is described by the variables Men
and Women, which are later combined into a percentage of women
(propWomen). Race and ethnicity are represented by variables Hispanic,
White, Black, Native, Asian, and Pacific in per cent. Finally, the
AvgAge represents the mean age of residents.

Economic and labor-market conditions are represented by several blocks
of variables. Unemployment variable represents the percentage of the
labor force that is unemployed. Educational block constitutes variables
that are summarized by the percentages of adults with less than a high
school diploma (LessThanHighSchool), a high school diploma
(HighSchoolDiploma), some college or an associate degree
(SomeCollegeOrAssociateDegree), and a bachelor's degree or higher
(BachelorDegreeOrHigher). Employment sectors variables include the
percentages of employed residents working in Professional, Service,
Office, Construction, and Production jobs. The type of employer block
includes variables that are the percentages of involvement in private
industry (PrivateWork), public jobs (PublicWork), self-employment
(SelfEmployed), and unpaid family work (FamilyWork). Transportation to
work is described by variables representing the percentages of workers
who Drive, Carpool, use Transit, Walk, use other means (OtherTransp), or
work from home (WorkAtHome). MeanCommute variable represents the mean
commute time in minutes. Together, these blocks provide a structured
description of county-level context for the exploratory analyses and
regression models that follow.

\subsection{Preprocessing and Missing
Data}\label{preprocessing-and-missing-data}

Before conducting exploratory analyses and fitting regression models, I
carried out basic preprocessing. All variables defined as percentages
were retained on their original 0--100 scale so that a 1-unit change
corresponds to a 1--percentage-point change. To summarize gender
composition, I constructed a single predictor,

\[\text{propWomen}_i = 100 \times \frac{\text{Women}_i}{\text{TotalPop}_i},\]

where \(\text{Women}_i\) and \(\text{TotalPop}_i\) denote the number of
women and total population in county \(i\). The percentage of men is
then implied as \(100 - \text{propWomen}_i\) and is not included
separately in the models, avoiding colinearity.

Several other predictor blocks also form compositions whose components
sum to approximately 100\%: race/ethnicity (Hispanic, White, Black,
Native, Asian, Pacific), employment sectors (Professional, Service,
Office, Construction, Production), commuting modes (Drive, Carpool,
Transit, Walk, OtherTransp, WorkAtHome), type of work or employer
(PrivateWork, PublicWork, SelfEmployed, FamilyWork), and educational
attainment (LessThanHighSchool, HighSchoolDiploma,
SomeCollegeOrAssociateDegree, BachelorDegreeOrHigher). For counties with
complete information, I checked the overall composition and they they
satisfy the following constraint:

\[\text{Hispanic}_i + \text{White}_i + \text{Black}_i + \text{Native}_i + \text{Asian}_i + \text{Pacific}_i \approx 100.\]

To avoid perfect collinearity in models, I designated one category in
each block as a reference and omitted it from the design matrix: White
for race/ethnicity, Professional for employment sectors, Drive for
transportation modes, PrivateWork for type of employer, and
LessThanHighSchool for educational attainment. Coefficients for the
included categories are therefore interpreted as changes in poverty (in
percentage points) associated with a 1--percentage-point increase in
that category, holding other predictors and the implied reference
category constant.

I examined missingness in the data and found that each of the four
education variables had eight missing values, corresponding to eight
counties with no recorded information on educational attainment.
Furthermore, AvgAge had five missing values, and the geographic
identifiers Long, Lat, and county--fips each had two missing values. For
exploratory plots, I included all available counties, including those
with missing data, so that the EDA reflects the full dataset. For
regression models, I excluded those 12 counties with missing data and
used complete-case analysis with respect to the outcome and the
predictors included in each model. Finally, I verified that all
percentage variables in the retained modeling dataset fall within the
valid range of 0 to 100\%; no negative or greater-than-100 values were
observed. A brief table listing the eight counties with missing
education data and their patterns of missingness is provided in Appendix
A.

\subsection{Exploratory Data Analysis}\label{exploratory-data-analysis}

\begin{figure}[H]

\centering{

\includegraphics[width=0.49\linewidth,height=\textheight,keepaspectratio]{poverty_density.png}

}

\caption{\label{fig-poverty-density}Density plot of county poverty rates
(Poverty, in percent).}

\end{figure}%

I first examined the distribution of county poverty rates. The
distribution of Poverty is moderately right-skewed: most counties fall
between roughly 10\% and 25\% poverty, with a long right tail including
a small number of counties above 40\%. This is shown in the density plot
in Figure 1. Boxplots of poverty by state (Appendix B, Figure B1)
indicate substantial between-state differences in both central tendency
and spread, with some states concentrated at relatively low poverty and
others centered at much higher levels.

Next, I explored economic and education variables. Counties with higher
unemployment rates tend to have substantially higher poverty, and the
smoothed curve in Figure 2 is clearly upward-sloping. Educational block
showed a similarly strong gradient. Counties with higher percentages of
adults holding a bachelor's degree or higher almost always have lower
poverty, whereas counties with higher percentages of adults with less
than a high school diploma have markedly higher poverty. The
intermediate categories (high school diploma and some college or an
associate degree) show weaker and more curved relationships as
illustrated in Figure 3. Average age and mean commute time have only
modest and slightly nonlinear associations with poverty (Appendix B,
Figure B2--B3). Overall, these results suggest that unemployment and
education composition are core predictors for RQ1, with age and commute
time playing more secondary roles.

\begin{figure}[H]

\centering{

\includegraphics[width=0.4\linewidth,height=\textheight,keepaspectratio]{poverty_vs_unemployment.png}

}

\caption{\label{fig-poverty-unemployment}County poverty rate versus
unemployment percentage.}

\end{figure}%

I then examined demographic and work-related variables. The derived
percentage of women, propWomen, is concentrated near 50\% in most
counties and has a very weak bivariate association with poverty: the
fitted curve is nearly flat, and the point cloud is dense and vertically
oriented (Appendix B, Figure B4). When looking into California and
Texas, separate fitted lines for each state as indicated in Figure 4,
remain shallow and close to horizontal, suggesting that any
state-specific differences in the propWomen--poverty association are
likely small. In contrast, racial and ethnic composition shows much
clearer patterns (Appendix B, Figure B5). Employment sector, work-type,
and commuting variables show similar gradients (Appendix B, Figures
B6--B7). Together, these exploratory results motivate including
unemployment, education, location, race/ethnicity, and selected
employment and work-type variables as primary predictors for RQ1, while
retaining propWomen to address RQ2 explicitly.

\begin{figure}[H]

\centering{

\includegraphics[width=0.6\linewidth,height=\textheight,keepaspectratio]{poverty_vs_education.png}

}

\caption{\label{fig-poverty-education}County poverty rate versus
percentages in each education category: less than high school, high
school diploma, some college or associate degree, and bachelor's degree
or higher.}

\end{figure}%

\begin{figure}[H]

\centering{

\includegraphics[width=0.5\linewidth,height=\textheight,keepaspectratio]{ca_vs_tx.png}

}

\caption{\label{fig-poverty-catx}County poverty rate versus percentage
women propWomen in California and Texas, with separate fitted lines by
state.}

\end{figure}%

\subsection{Modeling Strategy}\label{modeling-strategy}

To address RQ1, I modeled county poverty as a function of the main
demographic, geographic, and socioeconomic blocks described above. Let
\(\text{Poverty}_i\) denote the poverty rate (in percent) for county
\(i\). The primary model is a multiple linear regression: \[
\text{Poverty}_i= \beta_0 + \beta_1 \text{propWomen}_i + \beta_2 \text{Unemployment}_i + \mathbf{x}_i^\top \mathbf{\beta} + \varepsilon_i,
\]

where \(\text{propWomen}_i\) is the percentage of women,
\(\text{Unemployment}_i\) is the county unemployment rate, and
\(\mathbf{x}_i\) represents the remaining predictors from the
race/ethnicity, education, employment sector, commuting mode, work-type,
age, commute-time, and geographic blocks. Within some compositional
blocks, one category is omitted as a reference (White, Professional,
Drive, PrivateWork, LessThanHighSchool), so coefficients for the
included categories can be interpreted as changes in poverty associated
with a 1--percentage-point increase in that category, holding other
predictors constant. The exclusion also helps to avoid multicolinearity.

For RQ2, I restricted the data to counties in California and Texas and
examined whether the association between gender composition and poverty
differs by state. State was dummy coded with California as the reference
category (\(\text{StateTX}_i = 1\) for Texas, \(0\) for California), so
that the StateTX coefficient represents how average poverty in Texas
differs from California. I also created a mean-centered version of
gender composition,
\(\text{propWomen}_{c,i} = \text{propWomen}_i - \overline{\text{propWomen}}\),
so that \(\text{propWomen}_{c,i} = 0\) corresponds to the average
percentage of women across CA and TX. This makes the intercept and the
StateTX coefficient easier to interpret because they refer to a typical
county rather than to the unrealistic case of 0\% women. \[
\text{Poverty}_i = \beta_0 + \beta_1 \text{propWomen}_{c,i} + \beta_2 \text{StateTX}_i + \beta_3 \bigl(\text{propWomen}_{c,i} \times \text{StateTX}_i\bigr) + \mathbf{z}_i^\top \mathbf{\gamma} + \varepsilon_i,
\]

Here, \(\mathbf{z}_i\) represents a reduced set of control variables
that include unemployment, the education block (HighSchoolDiploma,
SomeCollegeOrAssociateDegree, BachelorDegreeOrHigher), and
race/ethnicity (Hispanic, Black, Native, Asian, and Pacific). I included
them because they seemed to have a strong association with poverty in
EDA and can be plausible confounders. Hence, adjusting for them may help
to isolate the association of propWomen\_c and its interaction with
StateTX.

In the model, \(\beta_0\) is the mean poverty rate for a California
county at the average percentage of women, \(\beta_1\) is the slope
relating gender composition to poverty in California, \(\beta_2\) is the
difference in baseline poverty between Texas and California at the
average proportion of women, and \(\beta_3\) is the difference in slopes
between Texas and California. The simple slope for Texas is therefore
\(\beta_1 + \beta_3\), so in the results I report both the regression
coefficients (propWomen\(_c\), StateTX, and their interaction) and
derived simple slopes for California (slope \(= \beta_1\)) and Texas
(slope \(= \beta_1 + \beta_3\)), each with heteroskedasticity-robust
confidence intervals.

For both models, I used coefficients from ordinary least squares (OLS)
and reported as unstandardized percentage-point units. I found
heteroskedasticity and heavy-tailed residuals while testing assumptions
and diagnosis, which I will discuss in the next section. Therefore, I
based all hypothesis tests and confidence intervals on
heteroskedasticity-robust (HC3; MacKinnon \& White, 1985) standard
errors. For RQ1, I treated the predictors as a single family and applied
the Benjamini--Hochberg false discovery rate procedure (Benjamini \&
Hochberg, 1995) at \(q = .05\) to the robust \(p\)-values. For RQ2, I
focused on a small number of pre-specified coefficients (the main effect
of gender composition and its interaction with state) and therefore
report robust \(p\)-values without additional multiple-testing
correction.

\subsection{Assumptions and Diagnosis}\label{assumptions-and-diagnosis}

\begin{figure}[H]

\centering{

\includegraphics[width=0.5\linewidth,height=\textheight,keepaspectratio]{resfit_lm.png}

}

\caption{\label{fig-residual-fitted}Residuals versus fitted values for
the RQ1 model.}

\end{figure}%

For the main RQ1 model, I examined linearity, homoskedasticity,
normality, independence, outliers, and multicollinearity using standard
regression diagnostics. The residuals versus fitted values plot (Figure
5) shows residuals centered around zero across the fitted range, with no
strong curvature, suggesting that the model is fairly linear. However,
the vertical spread of residuals increases somewhat for higher fitted
poverty values, producing a mild fan shape. This indicated
heteroskedasticity, and due to this, I based all hypothesis tests and
confidence intervals on heteroskedasticity robust (HC3; MacKinnon \&
White, 1985) standard errors for both RQ1 and RQ2.

I assessed normality of residuals using a histogram (Figure 6) and a
normal Q--Q plot (Appendix C, Figure C1). The histogram is roughly
symmetric and centered at zero, but the tails seem heavier than a normal
distribution. Given the large sample size and the use of robust standard
errors, I did not perform any transformations.

Additional diagnostics are reported in Appendix C. Residual maps and
residuals by state (Appendix C, Figures C2--C3) show only mild spatial
clustering and no large regions of systematic over or underprediction.
Plots of standardized residuals versus leverage and Cook's distance
(Appendix C, Figures C4--C5) indicate that only a small number of
observations have large residuals or moderate leverage, and no single
county has an undue influence on the fit. Variance inflation factors
(VIF) (Appendix C, Table C1) are mostly in the low to moderate range,
with higher values confined to the education block. This makes sense
because if one education category goes up then others should come down.
Overall, these diagnostics support using linear models on the original
poverty scale combined with robust standard errors and cautious
interpretation of tail behavior and spatial dependence.

\begin{figure}[H]

\centering{

\includegraphics[width=0.55\linewidth,height=\textheight,keepaspectratio]{norm_res_hist.png}

}

\caption{\label{fig-residual-hist}Histogram of residuals for the main
RQ1 regression model.}

\end{figure}%

\subsection{Multiple Testing}\label{multiple-testing}

Because the main RQ1 model includes a large set of predictors, relying
only on raw \(p\)-values would increase the chance of false positives.
To control false discoveries while retaining reasonable power, I treated
the main RQ1 predictors as a single family of hypotheses and applied the
Benjamini--Hochberg false discovery rate (FDR) procedure at \(q = .05\)
(Benjamini \& Hochberg, 1995). Let \(p_{(1)} \le \cdots \le p_{(m)}\)
denote the ordered \(p\)-values for the \(m\) substantive coefficients.
The Benjamini--Hochberg procedure finds the largest index \(k\) such
that \[
p_{(k)} \le \frac{k}{m} q,
\] and declares all hypotheses with \(p_{(j)} \le p_{(k)}\) as
FDR-significant. In the RQ1 results table (Table 1), I report robust
\(p\)-values for all coefficients and indicate which predictors remain
significant after FDR control at \(q = .05\). My inference about strong
and weak predictors is based on this FDR-adjusted \(p\)-values. For RQ2,
I examined only a small number of variables, so I did not apply a
multiple testing correction and interpreted the robust \(p\)-values
directly.

\section{Results}\label{results}

\subsection{RQ1: Associations Between Poverty and County
Characteristics}\label{rq1-associations-between-poverty-and-county-characteristics}

Table 1 reports the multiple regression model for RQ1, with Poverty as
the outcome and the full set of demographic, socioeconomic, and
geographic predictors. Coefficients are estimated by OLS with HC3 robust
standard errors; for inference on the main predictors, I have used
Benjamini--Hochberg adjusted \(p\)-values.

Unemployment and the educational blocks are the strongest predictors of
Poverty. Higher Unemployment is very strongly associated with higher
Poverty, whereas longer MeanCommute is associated with lower Poverty.
Relative to LessThanHighSchool, higher percentages with
HighSchoolDiploma, SomeCollegeOrAssociateDegree, and
BachelorDegreeOrHigher are each strongly associated with lower Poverty,
with SomeCollegeOrAssociateDegree showing the largest and most negative
coefficient in this block.

\begin{table}[htbp]
\centering
\caption{RQ1: County Poverty Regressed on Demographic, Geographic, and Socioeconomic Predictors}
\label{tab:rq1}
\includegraphics[width=\linewidth]{rq1_reg_table.png}
\end{table}

Race and ethnicity also show important patterns. Relative to White,
higher percentages of Black and Native residents are associated with
higher Poverty, while higher Hispanic and Asian percentages are
associated with lower Poverty. Pacific has a relatively large negative
point estimate but does not remain significant after FDR adjustment. In
the employment-sector block with Professional as the reference, Service
is a strong positive predictor of Poverty, whereas Office, Construction,
and Production have smaller coefficients are not statistically
significant. For type of employer with PrivateWork as reference,
PublicWork and SelfEmployed are moderately and significantly positively
associated with Poverty, while FamilyWork has the largest positive
estimate but is not FDR significant.

The contribution of transportation and remaining demographic/geographic
variables remain strong to modest. Relative to Drive, greater use of
Transit is positively associated with Poverty, whereas OtherTransp is
negatively associated with Poverty; Carpool, Walk, and WorkAtHome have
small and insignificant effects. The gender composition variable
propWomen is a strong positive predictor, indicating that counties with
a higher percentage of women tend to have higher Poverty, even after
adjustment. AvgAge is also a strong negative predictor, with older
counties having lower Poverty. Finally, the geographic coordinates Long
and Lat both have small but significant negative associations with
Poverty, while TotalPop has an essentially zero and nonsignificant
coefficient, making it the weakest predictor overall.

\subsection{RQ2: Gender Composition in California
vs.~Texas}\label{rq2-gender-composition-in-california-vs.-texas}

RQ2 tests whether the association between gender composition (propWomen)
and Poverty differs between California and Texas. As described before,
this model uses only counties in CA and TX, codes StateTX as 0 for
California and 1 for Texas, and relies on a mean-centered percentage of
women (propWomen\_c). The regression includes propWomen\_c, StateTX,
their interaction propWomen\_c:StateTX, and controls for Unemployment,
HighSchoolDiploma, SomeCollegeOrAssociateDegree, BachelorDegreeOrHigher,
Hispanic, Black, Native, Asian, and Pacific, with HC3 robust standard
errors. Table 2 presents the raw interaction model.

\begin{table}[htbp]
  \centering
  \caption{County Poverty Regressed on Gender Composition, State, and Controls}
  \label{tab:rq2_raw}
  \includegraphics[width=\linewidth]{rq2_rawreg_table.png}
\end{table}

Here, the coefficient on propWomen\_c gives the slope relating gender
composition to Poverty in California (the reference state) at the
average percentage of women; the StateTX coefficient gives the
difference in baseline Poverty between Texas and California at that
average; and the propWomen\_c:StateTX coefficient gives the difference
between the Texas and California slopes
(\(\text{slope}_{\text{TX}} - \text{slope}_{\text{CA}}\)). In Table 2,
none of these three coefficients is statistically significant.

To make the state-specific slopes explicit, I computed simple slopes for
California and Texas. Let \(\beta_1\) denote the coefficient on
propWomen\_c and \(\beta_3\) the coefficient on propWomen\_c:StateTX.
Then: \[
\text{slope}_{\text{CA}} = \beta_1,
\qquad
\text{slope}_{\text{TX}} = \beta_1 + \beta_3.
\] The variance of the Texas slope is: \[
\mathrm{Var}(\text{slope}_{\text{TX}}) = \mathrm{Var}(\beta_1) + \mathrm{Var}(\beta_3) + 2\,\mathrm{Cov}(\beta_1,\beta_3),
\] so that \[
\mathrm{SE}(\text{slope}_{\text{TX}}) = \sqrt{\mathrm{Var}(\text{slope}_{\text{TX}})}, \quad
t = \frac{\text{slope}_{\text{TX}}}{\mathrm{SE}(\text{slope}_{\text{TX}})}, \quad
\text{CI}_{95\%} = \text{slope}_{\text{TX}} \pm t_{.975}\,\mathrm{SE}(\text{slope}_{\text{TX}}).
\]

\begin{table}[htbp]
  \centering
  \caption{RQ2: Simple Slopes and Interaction of Gender Composition by State and Controls}
  \label{tab:rq2_tidy}
  \includegraphics[width=\linewidth]{rq2_tidyreg_table.png}
\end{table}

Table 3 reports these simple slopes together with the same controls as
in Table 2. As noted before, the simple slope for propWomen\_c in
California (CA) is positive but not statistically significant,
indicating little evidence of a gender--poverty association in
California. The slope for Texas (TX) is larger and statistically
significant, suggesting that in Texas, counties with more women tend to
have higher Poverty when unemployment, education, and race/ethnicity are
held constant. However, like before the difference between these slopes
(the CA:TX interaction) is not significant, indicating no strong
evidence that the propWomen--Poverty relationship truly differs between
California and Texas.

\section{Discussion and Conclusion}\label{discussion-and-conclusion}

For RQ1, the multiple regression shows that county level poverty is most
strongly associated with unemployment, education, race/ethnicity, gender
composition, and average age. Counties with higher unemployment and a
larger share of adults without a high school diploma tend to have
substantially higher poverty, whereas higher levels of educational
attainment are linked to notably lower poverty. Racial and ethnic
composition also matters: counties with larger Black and Native
populations have higher poverty, whereas those with larger Hispanic and
Asian populations tend to have lower poverty, net of the other
variables. Poverty is additionally higher in counties with more women
and lower in counties with older average age, with labor-market
(service, public sector, and self-employment) and commuting patterns
(greater public transit use) adding smaller but meaningful
contributions. Population size and several other employment and
transportation categories show little predictive value. Overall, these
findings are consistent with the EDA, which already highlighted
unemployment, education, and race/ethnicity as the clearest indicators
of county poverty.

For RQ2, the California--Texas interaction suggests that gender
composition is only weakly related to county poverty. The slope for the
percentage of women is small and not significant in California, and
somewhat larger and significant in Texas. This indicates that Texas
counties with more women tend to have higher poverty after controlling
for unemployment, education, and racial/ethnic blocks. However, the
difference between the California and Texas slopes is not itself
significant, so there is little evidence that the gender--poverty
relationship truly differs between the two states. This pattern is
consistent with the EDA, where the fitted lines for California and Texas
were closely overlapping (see Figure 4).

Substantively, the results suggest that county poverty is driven
primarily by local labor-market conditions, educational attainment, and
racial/ethnic composition, with gender composition and many detailed
employment and commuting measures playing a more limited role. The
analysis is cross-sectional and observational, so the associations
should not be interpreted as causal, and residual plots (Appendix B)
indicate some remaining spatial structure, suggesting that a full
spatial level modeling could provide a better account of geographic
dependence. Within these limits, the findings highlight the importance
of policies that reduce unemployment and expand educational
opportunities as drivers for lowering county-level poverty, and they
point to future work that incorporates richer spatial modeling,
additional policy covariates, and longitudinal data to study changes in
poverty over time.

I used ChatGPT (GPT-5.1 Thinking; OpenAI, 2025) to help me with R, debug
code, and polish the overall analysis. I designed the analysis plan by
myself, and then asked specific questions to ChatGPT mainly for sanity
checks. I asked whether particular modeling choices were appropriate and
what problems I might run into, while I implemented the code and
verified all results on my own.

\newpage

\section{References}\label{references}

Benjamini, Y., \& Hochberg, Y. (1995). Controlling the false discovery
rate: A practical and powerful approach to multiple testing.
\textit{Journal of the Royal Statistical Society: Series B (Methodological)},
\textit{57}(1), 289--300.

MacKinnon, J. G., \& White, H. (1985). Some
heteroskedasticity-consistent covariance matrix estimators with improved
finite sample properties. \textit{Journal of Econometrics},
\textit{29}(3), 305--325.

OpenAI. (2025, April 29).
\textit{GPT-5.1: Advancing reasoning and assistance}. Retrieved December
8, 2025, from \url{https://openai.com/index/gpt-5-1/}

\newpage
\appendix

\section{Appendix A: Counties Excluded From Regression
Models}\label{appendix-a-counties-excluded-from-regression-models}

\begin{figure}[H]
\centering
\includegraphics[width=0.75\textwidth]{appendixA_missingdata_table.png}
\caption{Counties Excluded From Regression Models Due to Missing Data}
\label{fig:A1}
\end{figure}

\newpage

\section{Appendix B: Exploratory Data
Analysis}\label{appendix-b-exploratory-data-analysis}

\begin{figure}[H]
\centering
\includegraphics[width=0.8\textwidth]{poverty_state_boxplot.png}
\caption{County Poverty Rates by State}
\label{fig:B1}
\end{figure}

\vspace{0.5cm}

\begin{figure}[H]
\centering
\includegraphics[width=0.8\textwidth]{poverty_vs_aveage_meancommute.png}
\caption{County Poverty Versus Average Age and Mean Commute Time}
\label{fig:B2}
\end{figure}

\vspace{0.5cm}

\begin{figure}[H]
\centering
\includegraphics[width=0.8\textwidth]{poverty_vs_propwomen.png}
\caption{County Poverty Versus Percentage Women}
\label{fig:B3}
\end{figure}

\vspace{0.5cm}

\begin{figure}[H]
\centering
\includegraphics[width=0.8\textwidth]{poverty_vs_race_ethnicity.png}
\caption{County Poverty Versus Race/Ethnicity Composition}
\label{fig:B4}
\end{figure}

\vspace{0.5cm}

\begin{figure}[H]
\centering
\includegraphics[width=0.8\textwidth]{poverty_vs_work_type.png}
\caption{County Poverty Versus Work Type Categories}
\label{fig:B5}
\end{figure}

\vspace{0.5cm}

\begin{figure}[H]
\centering
\includegraphics[width=0.8\textwidth]{poverty_vs_transportation.png}
\caption{County Poverty Versus Commuting Modes}
\label{fig:B6}
\end{figure}

\newpage

\section{Appendix C: Assumptions and
Diagnostics}\label{appendix-c-assumptions-and-diagnostics}

\begin{figure}[H]
\centering
\includegraphics[width=0.8\textwidth]{norm_res_qq.png}
\caption{Normal Q--Q Plot of Residuals for the Main Regression Model}
\label{fig:C1}
\end{figure}

\vspace{0.5cm}

\begin{figure}[H]
\centering
\includegraphics[width=0.8\textwidth]{resid_cor_latlong.png}
\caption{Spatial Pattern of Residuals by Longitude and Latitude}
\label{fig:C2}
\end{figure}

\vspace{0.5cm}

\begin{figure}[H]
\centering
\includegraphics[width=0.8\textwidth]{resid_cor_state.png}
\caption{Residuals by State}
\label{fig:C3}
\end{figure}

\vspace{0.5cm}

\begin{figure}[H]
\centering
\includegraphics[width=0.8\textwidth]{reslevarage_outliers.png}
\caption{Standardized Residuals Versus Leverage}
\label{fig:C4}
\end{figure}

\vspace{0.5cm}

\begin{figure}[H]
\centering
\includegraphics[width=0.8\textwidth]{cooksd_outliers.png}
\caption{Cook's Distance Values for Influential Counties}
\label{fig:C5}
\end{figure}

\vspace{0.5cm}

\begin{table}[H]
\centering
\includegraphics[width=0.60\textwidth]{appendixB_vif_table.png}
\caption{Variance Inflation Factors for Predictors in the Main Regression Model}
\label{table:C1}
\end{table}




\end{document}
